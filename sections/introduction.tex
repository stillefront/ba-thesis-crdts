Im Rahmen der vorliegenden Arbeit, soll die Entwicklung eines kollaborativen Vektorzeichneprogramms
dargelegt werden. 
Gegenentwurf zu moderner, verbreiteter Software, die als SaaS funktioniert und Dokumente und
Dateien immer weniger offline verfügbar sind und erst \glqq{}exportiert\grqq{} werden müssen, möchte ich aufzeigen, dass es möglich sei, ein Vektorzeichenprogramm umzusetzen, dass in der Lage ist, mehrere Nutzer gleichzeitig an einem Dokument arbeiten zu lassen und ohne zentrale Serverinstanz auskommt. 
Nutzer sollen in der Lage sein, ohne zentralen Server zur Datenspeicherung und Konfliktlösung bei gleichzeitiger Bearbeitung an einem Grafikdokument zu arbeiten. 
Natürlich existieren zum gegenwärtigen Zeitpunkt Anwendungen im Bereich von Texteditoren und einfachen Whiteboards, sowohl im kommerziellen als auch im Open-Source-Umfeld [zitieren welche Software gemeint ist], jedoch ist mir keine Grafikanwendung bekannt, die das mir vorschwebende Nutzungsszenario abbildet und dabei auf eine reine Peer-to-Peer-Architektur setzt.\footnote{Hier sollte eine Fußnote erscheinen äöüá}
\par 
Mein Gegenentwurf soll aufzeigen, dass man kollaborative Kreativsoftware ohne zentrale Instanz erstellen
können kann, die die Privatsphäre von Nutzern und einem erhöhten Anspruch an Datenschutz gerecht werden kann. 
Kollaboration sollte auch ohne zentrale Serverinstanzen möglich sein und.

\begin{lstlisting}[caption={Bsp.: Hello World (React)}]
const HelloWorld = () => {
	let blah = 123;
	let blubb = "blubb";
	return (
		<div> Hello World </div>
	)
}

export default HelloWorld;
\end{lstlisting}




Citing stuff \cite{kleppmanLocalfirstSoftwareYou2019}
