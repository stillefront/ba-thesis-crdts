% MAIN STUFF
\documentclass[oneside,bibliography=totocnumbered,BCOR=5mm]{scrbook}
\usepackage{amsmath, amsthm, amssymb}
\usepackage[ngerman]{babel}
\usepackage{graphics}
\usepackage{csquotes}
\usepackage{setspace}

% BIBLIOGRAPHY
\usepackage[backend=biber]{biblatex}
\addbibresource{bibliography/references.bib}
\usepackage{titling} % customize title page
\usepackage[hmarginratio=1:1,top=32mm,columnsep=20pt]{geometry} 
\usepackage[hang, small,labelfont=bf,up,textfont=it,up]{caption} 
\usepackage{booktabs} 
\usepackage{lettrine} 
\usepackage{enumitem}
\setlist[itemize]{noitemsep}

% LISTINGS

\usepackage{listings}
\usepackage{color}


\lstdefinelanguage[ECMAScript2015]{JavaScript}[]{JavaScript}{
  morekeywords=[1]{await, async, case, catch, class, const, default, do,
    enum, export, extends, finally, from, implements, import, instanceof,
    let, static, super, switch, throw, try},
  morestring=[b]` % Interpolation strings.
}


\lstdefinelanguage{JavaScript}{
  morekeywords=[1]{break, continue, delete, else, for, function, if, in, let,
    new, return, this, typeof, var, void, while, with},
  % Literals, primitive types, and reference types.
  morekeywords=[2]{false, null, true, boolean, number, undefined,
    Array, Boolean, Date, Math, Number, String, Object},
  % Built-ins.
  morekeywords=[3]{eval, parseInt, parseFloat, escape, unescape},
  sensitive,
  morecomment=[s]{/*}{*/},
  morecomment=[l]//,
  morecomment=[s]{/**}{*/}, % JavaDoc style comments
  morestring=[b]',
  morestring=[b]"
}[keywords, comments, strings]

\lstalias[]{ES6}[ECMAScript2015]{JavaScript}



\definecolor{mygreen}{rgb}{0,0.6,0}
\definecolor{mygray}{rgb}{0.5,0.5,0.5}
\definecolor{mymauve}{rgb}{0.58,0,0.82}

\lstset{ 
  backgroundcolor=\color{white},   % choose the background color; you must add \usepackage{color} or \usepackage{xcolor}; should come as last argument
  basicstyle=\ttfamily,        % the size of the fonts that are used for the code
  breakatwhitespace=false,         % sets if automatic breaks should only happen at whitespace
  breaklines=true,                 % sets automatic line breaking
  captionpos=b,                    % sets the caption-position to bottom
  commentstyle=\color{mygreen},    % comment style
  deletekeywords={...},            % if you want to delete keywords from the given language
  escapeinside={\%*}{*)},          % if you want to add LaTeX within your code
  extendedchars=true,              % lets you use non-ASCII characters; for 8-bits encodings only, does not work with UTF-8
  firstnumber=1,                % start line enumeration with line 1000
  frame=single,	                   % adds a frame around the code
  keepspaces=true,                 % keeps spaces in text, useful for keeping indentation of code (possibly needs columns=flexible)
  keywordstyle=\color{blue},       % keyword style
  language=ES6,         % the language of the code
  morekeywords={*,...},            % if you want to add more keywords to the set
  numbers=left,                    % where to put the line-numbers; possible values are (none, left, right)
  numbersep=5pt,                   % how far the line-numbers are from the code
  numberstyle=\tiny\color{mygray}, % the style that is used for the line-numbers
  rulecolor=\color{black},         % if not set, the frame-color may be changed on line-breaks within not-black text (e.g. comments (green here))
  showspaces=false,                % show spaces everywhere adding particular underscores; it overrides 'showstringspaces'
  showstringspaces=false,          % underline spaces within strings only
  showtabs=false,                  % show tabs within strings adding particular underscores
  stepnumber=1,                    % the step between two line-numbers. If it's 1, each line will be numbered
  stringstyle=\color{mymauve},     % string literal style
  tabsize=2,	                   % sets default tabsize to 2 spaces
  title=\lstname                   % show the filename of files included with \lstinputlisting; also try caption instead of title
}


\begin{document}
\begin{titlepage}
\begin{center}
\includegraphics{HTW_Berlin_Logo_farbig.jpg}
\linebreak[4]
\linebreak[4]
\linebreak[4]
\linebreak[4]
\textit{\large Prototypische Entwicklung eines kollaborativen und dezentralisierten Vektorzeichenprogram auf Basis von Conflict-free Replicated Data Types (CRDTs)}
\linebreak[4]
\linebreak[4]
\linebreak[4]
Abschlussarbeit 
\linebreak[4]
\linebreak[4]
zur Erlangung des akademischen Grades 
\linebreak[4]
\linebreak[4]
\textbf{Bachelor of Science (B.Sc.)}
\linebreak[4]
\linebreak[4]
an der
\linebreak[4]
\linebreak[4]
Hochschule für Technik und Wirtschaft (HTW) Berlin
\linebreak[4]
Fachbereich 4: Informatik, Kommunikation und Wirtschaft
\linebreak[4]
Studiengang \textit{Internationale Medieninformatik}
\linebreak[4]
\linebreak[4]
\linebreak[4]
1. Gutachter: Prof. Dr. Gefei Zhang\linebreak[4]
2. Gutachter: Prof. Dr. Klaus Jung\linebreak[4]
\linebreak[4]
\linebreak[4]
\linebreak[4]
\linebreak[4]
Eingereicht von Alexej Bormatenkow 570108
\linebreak[4]
\linebreak[4]
\linebreak[4]
\linebreak[4]
\date{\today}


\end{center}
\end{titlepage}
\newpage
\thispagestyle{empty}
\newpage
\thispagestyle{empty}       % keine Seitennummer
\section*{Zusammenfassung}
[Text der Zusammenfassung]


\clearpage
%Seite 1
\pagenumbering{roman}% Seitennummerierung "roemisch"
%\setcounter{page}{1} 

\tableofcontents
.

\newpage
\pagenumbering{arabic}

\chapter{Einleitung und Motivation}
Im Rahmen der vorliegenden Arbeit, soll die Entwicklung eines kollaborativen Vektorzeichneprogramms
dargelegt werden. 
Gegenentwurf zu moderner, verbreiteter Software, die als SaaS funktioniert und Dokumente und
Dateien immer weniger offline verfügbar sind und erst \glqq{}exportiert\grqq{} werden müssen, möchte ich aufzeigen, dass es möglich sei, ein Vektorzeichenprogramm umzusetzen, dass in der Lage ist, mehrere Nutzer gleichzeitig an einem Dokument arbeiten zu lassen und ohne zentrale Serverinstanz auskommt. 
Nutzer sollen in der Lage sein, ohne zentralen Server zur Datenspeicherung und Konfliktlösung bei gleichzeitiger Bearbeitung an einem Grafikdokument zu arbeiten. 
Natürlich existieren zum gegenwärtigen Zeitpunkt Anwendungen im Bereich von Texteditoren und einfachen Whiteboards, sowohl im kommerziellen als auch im Open-Source-Umfeld [zitieren welche Software gemeint ist], jedoch ist mir keine Grafikanwendung bekannt, die das mir vorschwebende Nutzungsszenario abbildet und dabei auf eine reine Peer-to-Peer-Architektur setzt.\footnote{Hier sollte eine Fußnote erscheinen äöüá}
\par 
Mein Gegenentwurf soll aufzeigen, dass man kollaborative Kreativsoftware ohne zentrale Instanz erstellen
können kann, die die Privatsphäre von Nutzern und einem erhöhten Anspruch an Datenschutz gerecht werden kann. 
Kollaboration sollte auch ohne zentrale Serverinstanzen möglich sein und.

\begin{lstlisting}[caption={Bsp.: Hello World (React)}]
const HelloWorld = () => {
	let blah = 123;
	let blubb = "blubb";
	return (
		<div> Hello World </div>
	)
}

export default HelloWorld;
\end{lstlisting}




Citing stuff \cite{kleppmanLocalfirstSoftwareYou2019}


\newpage
\chapter{Kollaborative Software}
\input{sections/collaborative_software}

\section{Local-first Software}
There's no cloud, it's just someone else's computer




\section{Datenreplikation im Netzwerk}
\input{sections/data_replication}

\subsection{Replikationsstrategien}
\input{sections/replication_strategies}

\subsection{Replikationsmodelle}
\input{sections/replication_models}

\subsection{Operational Transformation (OT)}
\input{sections/operational_transformation}
\subsection{Vector Clocks}
\input{sections/vector_clocks}


\subsection{Conflict-free Replicated Data Types (CRDTs)}
\input{sections/crdts}
\subsubsection{State-based CRDTs}
\input{sections/state_based_crdts}
\subsubsection{Operation-based CRDTs}
\input{sections/operation_based_crdts}

\subsection{Datenreplikation in verteilten Systemen}
\input{sections/data_replication_distributed_systems}



\printbibliography[
heading=bibintoc,
title={Quellenverzeichnis}
]
\end{document}
